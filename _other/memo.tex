\documentclass{jsarticle}

\usepackage{okumacro}
\usepackage{amsmath,amssymb}

\newcommand{\abs}[1]{\left| #1 \right|}
\newcommand{\suffix}[2]{#1_\mathrm{#2}}
\newcommand{\mami}{\begin{center}ティロ・フィナーレ\end{center}}

\begin{document}

\part{ラインの黄金}
あいうえお
\paragraph{パラグラフ}
かきくけこ

Lucy in the Sky with (\ \ \ \ \ \ \ \ \ \ ).

\LaTeX{} is XXX!

\ruby{魔弾の舞踏}{ダンザデルマジックバレット} \par
\mami
\mami
\mami
\[ \abs{\dot{V}} = \sqrt{a^2+b^2} \]
\[ \suffix{V}{in} = 0 \]

\LaTeX の数式モードでは,イタリック体とは微妙に異なる.
$The difference story$と\textit{The Difference story}.\par

$2^10 = 1024$は間違いです.括弧で囲みましょう. $2^{10} = 1024$

次に,\AmS-\LaTeX を使う.
\begin{equation}
    A = \begin{pmatrix}
        a_{11} & a_{12} \\
        a_{21} & a_{22} \\
    \end{pmatrix}
\end{equation}
$ a_1, a_2 \dots a_n$

\end{document}

